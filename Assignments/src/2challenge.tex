\documentclass[11pt]{article}

%%  Dimensions and URL
\usepackage[margin=1in]{geometry}
\usepackage{hyperref}

%%  Definitions
\renewcommand{\baselinestretch}{1.1}
\pagestyle{plain}


\begin{document}

\begin{center}
{\bfseries \LARGE Programming Challenge 2}
\end{center}

This challenge, like many other challenges, will use the \texttt{random} module in python to generate pseudo random numbers.
Remember to import the module, if needed.
\begin{verbatim}
import random
\end{verbatim}

Write a method to generate a Bernoulli random variable with parameter $p = 0.75$.
Then, create a sequence of random variables.
Each random variable in the sequence should be the sum of 10 independent Bernoulli random variable.
Note that the range $X(\Omega)$ for each random variable should be $\{ 0, 1, \ldots, 10 \}$.
Using a \texttt{for} loop, look at the empirical distribution of the proportion of zeros, ones, \ldots, tens.
\begin{verbatim}
percent = []
for OutcomeIndex in range(0, SampleSpaceSize):
    percent.append(TrialSequence.count(OutcomeIndex) / float(NumberTrials))
print percent
\end{verbatim}
Explore how the empirical distribution changes as \texttt{N} grows: 100.0, 1000.0, 10000.0, etc.
Can you recognize the distribution that you are getting?

Use \texttt{matplotlib}, or another python 2D plotting library to display empirical distributions.

\end{document}

