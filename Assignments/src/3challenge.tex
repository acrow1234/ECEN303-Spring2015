\documentclass[11pt]{article}

%%  Dimensions and URL
\usepackage[margin=1in]{geometry}
\usepackage{hyperref}

%%  Definitions
\renewcommand{\baselinestretch}{1.1}
\pagestyle{plain}


\begin{document}

\begin{center}
{\bfseries \LARGE Programming Challenge 3}
\end{center}

Use the \texttt{random} module to generate pseudo random numbers.
For instance, \texttt{random.randrange(2)} produces (pseudo) random bits.
To use this module, it is necessary to \texttt{import random}.
It may also be helpful to \texttt{import numpy as np}.

In this numerical procedure, you will create multiple sequences of random variables.
Let \texttt{X} and \texttt{Y} correspond to the roll of independent, fair dice.
Let sum \texttt{S = X + Y} and max \texttt{M = max(X,Y)}.
Compute the joint PMF of \texttt{S} and \texttt{M}.
\begin{verbatim}
NumberTrials = 100000
sequenceX = []
sequenceY = []
sequenceS = []
sequenceM = []

for TrialIndex in range(0, NumberTrials):
    sequenceX.append(random.randint(1, 6))
    sequenceY.append(random.randint(1, 6))
    sequenceS.append(sequenceX[TrialIndex] + sequenceY[TrialIndex])
    sequenceM.append(max(sequenceX[TrialIndex], sequenceY[TrialIndex]))
\end{verbatim}
Then, look at the empirical distribution of the ratios of zeros and ones.
\begin{verbatim}
PMFofSM = np.zeros((13, 7))

for TrialIndex in range(0, NumberTrials):
    PMFofSM[sequenceS[TrialIndex], sequenceM[TrialIndex]] += 1

PMFofSM /= float(NumberTrials)
\end{verbatim}
Write code to isolate the (empirical) conditional PMF of \texttt{S} given \texttt{M}.
Explore how this empirical distribution changes as \texttt{N} increases: 10, 100, 1000, 10000.
Can you guess the correct structure for the conditional PMF of \texttt{S} given \texttt{M}?
\end{document}

