\documentclass[11pt]{article}

%%  Dimensions and URL
\usepackage[margin=1in]{geometry}
\usepackage{hyperref}

%%  Definitions
\renewcommand{\baselinestretch}{1.1}
\pagestyle{plain}


\begin{document}

\begin{center}
{\bfseries \LARGE Challenge 1}
\end{center}

\section*{Programming Challenges}

In python, the \texttt{random} module generates pseudo random numbers.
For instance, \texttt{random.randrange(2)} produces (pseudo) random bits.
To use this module, it is necessary to \texttt{import random}.

Using a loop, store $n$ random bits in a \texttt{list} object.
\begin{verbatim}
N = 100.0
sequence = []
for i in range(0, N):
    sequence.append(random.randrange(2))
\end{verbatim}
Then, look at the empirical distribution of the ratios of zeros and ones.
\begin{verbatim}
percent = []
percent.append(sequence.count(0) / N)
percent.append(sequence.count(1) / N)
print percent
\end{verbatim}
Explore how the empirical distribution changes as \texttt{N} increases 10.0, 100.0, 1000.0, 10000.0.

\end{document}
